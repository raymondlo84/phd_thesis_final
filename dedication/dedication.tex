I would like to express my deepest gratitude to Professor Steve Mann who inspired me to embark on my journey in wearable computing. I still vividly remember the first day I met with Professor Mann back in 2006. In his first lecture, he bought in many of his research equipments from his prior projects, and had given us a set of real-time live demonstrations. That day he told us that he can see beyond what human eye is possible. For example, we can observe physical phenomenas that are not visible to our bare human eye with his EyeTap, such as radio waves. Despite his popularity and busy schedule, he was very approachable, and he also introduced me to the core members from the research lab. At that time, he was wearing his EyeTap eyeglasses invention, and he was showing us the possibility of creating a digital eyeglasses that allows us to see better, and be able to download prescription from the internet. It is still mind blowing to be able to turn what we learn into something we can use everyday. I could never achieve what I have today without his unique guidance and giving me such opportunity in the first place.. 

I would like to thank Dr. James Fung who was my first mentor and the first providing me guidance for entering this amazing PhD program. We have collaborated on various projects such as cyborglogging and James had provided me the first exposure to parallel programming and real-time computer vision processing with GPUs. 

I would like to thank everyone who contributed to the research we collaborated on the HDR and Digital Eye Glass project. Specifically, I would like to thank Valmiki Rampersad who had bought the GPU version of the HDR project to life. For the FPGA collaboration efforts, I would like to thank Tao Ai and Kalin Ovtcharov who had made the proposed algorithm implementable on FPGA. For the wearable eyeglasses project, I would like to thank Han Wu who designed the first generation of the Digital Eye Glasses with me. Without his design, the wearable computers will never be able to take the final form. I'm grateful to have all these partners who believed in the vision. There were countless nights we all come together, and putting all efforts into a system which changes the way we see one day.
 
Lastly, I would like to thank Meron Gribetz who co-founded Meta with me and also provided the opportunity to run the company as the Chief Technology Officer. Meta is now a company with over a hundred of employees in 2017, and we have successfully invented and shipped two Augmented Reality Eyeglasses products to the market. This experience had bought me insight and core validation about what we had been pursuing for the last decade. He is a unique and strong person who can pushed my boundary and my capability in all dimensions.

Today, I believe that we will be wearing a pair eyeglasses or alike which will redefine the way we see this world. This vision is no longer my own dream, but a collective mind from everyone in the history.