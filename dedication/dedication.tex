I would like to express my deepest gratitude to Professor Steve Mann who inspired me to embark on my journey in wearable computing. I still vividly remember the first day when I met with Professor Mann back in 2006. In his first lecture, he brought in his research equipment from his prior projects, and gave us a set of real-time live demonstrations. That day, he told the class that he could see beyond the human eye. For example, he claimed that he could observe physical phenomena not visible to us with his EyeTap device, such as the WiFi signal. Despite his popularity and busy schedule, he has been very approachable, and he also introduced me to the core members from the research lab. At that time, he was wearing his EyeTap invention, and he was showing us the possibility of creating a digital form of eyeglasses that would allow us to see better (e.g., by ``downloading the prescription" from the internet). It is mind-blowing to be able to turn what we learn into something we can use everyday. I could have never achieved what I have today without his guidance and the unique opportunity that he provided in the first place. 

I would like to thank Dr. James Fung who provided me guidance on pursuing my PhD studies. We have collaborated on various projects such as cyborglogging and James showed me the path in parallel programming and real-time computer vision processing with GPUs. 

I would like to thank everyone who contributed to the various phases of the research efforts on HDR processing and the Digital Eye Glass project. Specifically, I would like to thank Valmiki Rampersad who assisted with bringing the GPU version of the HDR project to life. For the FPGA efforts, I would like to thank Tao Ai and Kalin Ovtcharov who made the proposed algorithm implementable on FPGAs. In addition, I would like to thank Han Wu who assisted with the design of the initial iteration of the Digital Eye Glass. I am grateful to have worked with so many talented people who believe in the same vision.
 
Finally, I would like to thank my family and Prof Mann who supported me during my entrepreneurial pursuit to translate our wearable technology to a commercial product at Meta (a technology start-up based in Silicon Valley) as Co-Founder, Co-Inventor and CTO. Meta has grown tremendously during my time there from a company with only a few individuals initially to now a company with over a hundred employees. We have successfully shipped two products, which have been recognized with numerous awards (including most recently the SIGGRAPH 2016 Best of the Show award) and have been featured extensively in news media globally.  This experience led to a number of unique opportunities, including being invited to speak (with live demos!) at leading international conferences and events such as the Hello Tomorrow global summit (Paris in 2016 and Korea in 2017), Techcracker global startup road show (in various cities in Asia organized by the Li Ka Shing Foundation) as well as the Tencent WE summit (Beijing in 2016).  In addition, this transformative journey gave me significant insights into starting up a business from scratch and provided important validation on the vision we have been pursuing in the lab for the past decade. In the future, I truly believe we will all be wearing a pair of digital eyeglasses that will dramatically transform the way we see this world. This vision is no longer just a dream of mine. 